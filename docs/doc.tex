\documentclass[12pt,a4paper]{article}

\usepackage[left=3cm, right=3cm, top=2.5cm, bottom=2.5cm, headheight=38.40865pt]{geometry} % Adjusted headheight
\usepackage{setspace}
\usepackage{amsmath}
\usepackage{tikz}
\usepackage{pgfplotstable}
\usepackage{titlesec}
\usepackage{bm}
\usepackage{tcolorbox}
\tcbuselibrary{skins}
\usepackage{empheq}
\usepackage{booktabs}
\usepackage{caption}
\usepackage{hyperref}
\usepackage{fancyhdr}
\usepackage{float}
\usepackage{silence}
\usepackage{multirow}
\usepackage{verbatim}
\WarningFilter{caption}{The option `hypcap=true' will be ignored}
\WarningFilter{latex}{Underfull \hbox}
\hypersetup{
    colorlinks=true,
    linkcolor=black,
    filecolor=magenta,      
    urlcolor=cyan,
    pdfpagemode=FullScreen,
    }
\usepackage{graphicx}
\graphicspath{ {./images/} }

\pgfplotsset{compat=1.18}

\titleformat{\section}{\Large\bfseries}{\thesection}{1em}{}
\titleformat{\subsection}{\large\bfseries}{\thesubsection}{1em}{}

\title{Interférences et diffraction de la lumière}
\author{Liviu Arsenescu, Cătălin Bozan}
\date{28.05.2024}

\newtcbox{\mymath}[1][]{%
    nobeforeafter,
    tcbox raise base,
    enhanced,
    colframe=black,
    colback=white,
    boxrule=1pt,
    drop shadow={
        xshift=3pt, % Removed 'shadow' prefix
        yshift=-3pt, % Removed 'shadow' prefix
        opacity=1
    },
    #1
}

\pagestyle{fancy}
\fancyhf{}
\rhead{\includegraphics[width=4cm]{hearclogo.png}}
\lhead{\thepage}
\setlength{\headsep}{30pt}

\begin{document}
    \pagenumbering{gobble}
    \begin{titlepage}
        \begin{center}
            \vspace*{\fill}
            \Huge \textbf{Client SMTP in C} \\
            \Large Instructions and documentations \\
            \begin{figure}[h]
                \centering
                \includegraphics[width=7cm]{hearclogo.png}
            \end{figure}
            \vspace{\fill}
            \Large Liviu Arsenescu \\
            16.06.2024

            \vspace*{\fill}
        \end{center}
    \end{titlepage}

    \tableofcontents
    \pagenumbering{arabic}
    \newpage

    \section{Introduction}
    Welcome to the documentation for the Simple SMTP Client, a project developed for the Networking course at HE-Arc Ingénierie. This project showcases the practical application of networking concepts by implementing a basic SMTP client, designed to facilitate the understanding of email protocols and client-server communication.
    \section{Client usage}
    \subsection{Synopsis}
    \begin{verbatim}
        bin/client_smtp
            <sender email>
            <subject>
            <message file>
            <mail server>
            <reciever email>
            [<port>]
    \end{verbatim}
    \subsection{Description}
    \verb|<sender email>| - Email address of the sender. \\
    \verb|<subject>| - Subject of the email. \\
    \verb|<message file>| - Path to the file containing the message. \\
    \verb|<mail server>| - Address of the mail server. \\
    \verb|<reciever email>| - Email address of the reciever. \\
    \verb|[<port>]| - (optional) Port number of the mail server. Default is 25.
    \subsection{Example}
    \begin{verbatim}
        bin/client_smtp
            liviu-andrei.arsenescu@he-arc.ch
            "Some Subject"
            mail_body.txt
            smtp.alphanet.ch
            liviu-andrei.arsenescu@he-arc.ch
            587
    \end{verbatim}
    \subsection{Compile the program}
    To compile the program, you can use the Makefile provided:
    \begin{verbatim}
        make
    \end{verbatim}
    The program executable is situated in \verb|./bin|
    \section{Implementation and State Machine}
    \section{Tests}
\end{document}
